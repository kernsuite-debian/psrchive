\documentclass[12pt]{article}
\usepackage{amsfonts}
\usepackage{bm}

% bold math italic font
\newcommand{\mbf}[1]{\mbox{\boldmath $#1$}}

% symbol used for sqrt(-1)
\newcommand{\Ci}{{\rm i}}

\newcommand{\C}{\mathbb{C}}
\newcommand{\R}{\mathbb{R}}

\newcommand{\var}{\rm Var}
\newcommand{\trace}{\rm Tr}

\newcommand{\code}[1]{{\tt{#1}}}

\newcommand{\rotation}[3]{{\ensuremath{ {\bf R}^{#3}_{#1}({#2}) }}}

\newcommand{\boost}[1][1]{{\ensuremath{ {\bf B}_{\bm{\hat {#1}}}(\gamma) }}}
\newcommand{\rotat}[1][1]{{\ensuremath{ {\bf R}_{\bm{\hat {#1}}}(\phi) }}}

\newcommand{\pauli}[1]{\ensuremath{ {\bm\sigma}_{\rm #1} }}
\newcommand{\para}[1][ ]{\ensuremath{ {\Phi}_{{\rm PA}{#1}} }}

\newcommand{\model}[1][ ]{\ensuremath{{\bm\rho}_j({{\bf T}_{#1}};{\mbf{a}}) }}
\newcommand{\obs}{\ensuremath{ {\bm\rho}_{j,i} }}

\begin{document}

\noindent
\large
{\sc psrchive} Implementation Notes \\ [2mm]
\normalsize
Willem van Straten \\
7 October 2013

\section*{Basis-independent Ideal Feed Assumption}

This document defines the equations used to derive the
polarimetric response based on the ideal feed assumption
(implemented by the {\tt SingleAxisSolver} class
in {\sc psrchive}).
%
In the ideal feed approximation to calibration, it is assumed that the
Jones matrix has the form
%
\[
{\bf J}=\left( \begin{array}{cc}
z_0 & 0 \\
0 & z_1
\end{array}\right)
\]
%
where $z_0$ and $z_1$ are the complex gains applied to each receptor.
%
This Jones matrix may be represented using its polar decomposition,
\[
{\bf J} = J \; \boost \rotat
\]
where $J=(\det{\bf J})^{1/2}$, \boost\ is a Hermitian matrix (or boost
transformation) and \rotat\ is a unitary matrix (or rotation
transformation).  Note that both the boost and rotation are parameterized
by a single unit vector, $\bm{\hat 1}$ (the {\it single axis}). 
%
The absolute phase of the signal is lost during detection; therefore,
the phase of the complex-valued $J$ may be arbitrarily chosen, and $J$
may be replaced by the real-valued gain, $G=|J|$.  The instrumental
response is then completely described by the absolute gain $G$,
differential gain $\gamma$, and differential phase $\phi$.

The boost transformation affects only the total intensity and the
component of the polarization vector that is parallel to $\bm{\hat 1}$,
leaving the component of the polarized flux that is perpendicular to
$\bm{\hat 1}$ unchanged.  Conversely, the rotation transformation
affects only the polarized flux perpendicular to $\bm{\hat 1}$, leaving
the component that is parallel to $\bm{\hat 1}$ and the total intensity
unchanged.

In principle, any set of input and output Stokes parameters can be
used to measure the ideal feed parameters, so long as there is
sufficient polarization perpendicular to $\bm{\hat 1}$ to constrain
the differential phase.  In practise, a 100\% linearly polarized signal is
generated using a noise diode that is coupled such that it induces an
in-phase response with equal amplitude in each receptor; see the
discussion of {\bf Calibrator Phase} in van Straten et al. (2010; PASA
27, 104) for more details.

\subsection*{Derivation of Solution}

Let $S=[S_0,\bm{S}]$ represent the expected (or input) Stokes
parameters of a known calibrator source and
$S^\prime=[S_0^\prime,\bm{S}^\prime]$ represent the observed (or
output) Stokes parameters.  It is useful to decompose both
polarization vectors into their components parallel and perpendicular
to the primary (single) axis; e.g.
\[
\bm{S_\parallel} = (\bm{S \cdot \hat 1}) \bm{\hat 1} = S_1\bm{\hat 1}
\]
\[
\bm{S_\perp} = \bm{S} - \bm{S_1} = S_2\bm{\hat 2} + S_3\bm{\hat 3}
\]
%
Then the cosine and sine of the angle of rotation about $\bm{\hat 1}$ between input and output
states are given by
%
\[
\bm{S_\perp \cdot S_\perp^\prime} = |\bm{S_\perp}| |\bm{S_\perp^\prime}| \cos2\phi
\]
\[
(\bm{S_\perp \times S_\perp^\prime}) \bm{\cdot \hat 1} = |\bm{S_\perp}| |\bm{S_\perp^\prime}| \sin2\phi
\]
such that
\begin{equation}
\tan2\phi = { {S_2 S_3^\prime - S_3 S_2^\prime} \over S_2 S_2^\prime + S_3 S_3^\prime  }
\end{equation}
Transformation by $G$\boost\ results in
\begin{eqnarray*}
S_0^\prime = G^2 \left( S_0 \cosh2\gamma + S_1 \sinh2\gamma \right) \\
S_1^\prime = G^2 \left( S_0 \sinh2\gamma + S_1 \cosh2\gamma \right)
\end{eqnarray*}
from which it is trivial to derive
\begin{equation}
\tanh2\gamma = { {S_0 S_1^\prime - S_1 S_0^\prime} \over S_0 S_0^\prime - S_1 S_1^\prime  }
\end{equation}
and
\begin{equation}
G^2 = { {S_0^\prime S_0^\prime - S_1^\prime S_1^\prime} \over S_0 S_0 - S_1 S_1 }
\end{equation}
\end{document}


